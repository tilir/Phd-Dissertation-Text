\chapter*{Заключение}                       % Заголовок
\addcontentsline{toc}{chapter}{Заключение}  % Добавляем его в оглавление

%% Согласно ГОСТ Р 7.0.11-2011:
%% 5.3.3 В заключении диссертации излагают итоги выполненного исследования, рекомендации, перспективы дальнейшей разработки темы.
%% 9.2.3 В заключении автореферата диссертации излагают итоги данного исследования, рекомендации и перспективы дальнейшей разработки темы.
%% Поэтому имеет смысл сделать эту часть общей и загрузить из одного файла в автореферат и в диссертацию:

Основные результаты работы заключаются в следующем.

\begin{enumerate}[beginpenalty=10000] % https://tex.stackexchange.com/a/476052/104425
  \item Разработана методология представления высокоуровневых векторных конструкций в векторной системе команд. Указанная методология применена к компилятору IGC.
  \item Разработан алгоритм разбиения структур данных для улучшения векторизации.
  \item Разработан алгоритм восстановления векторного потока управления.
  \item Получены результаты как по корректности так и по производительности кода.
\end{enumerate}

Таким образом, в диссертационной работе исследованы и разработаны методология и алгоритмы для повышения эффективности ключевых приложений в гетерогенных системах, что имеет существенное значение для области вычислений на графических ускорителях.

Перспективы дальнейшей разработки темы данной диссертации не ограничиваются графическими системами. В данный момент автор работает над масштабированием своих идей на кластеры RISCV процессоров в том числе с векторными расширениями.